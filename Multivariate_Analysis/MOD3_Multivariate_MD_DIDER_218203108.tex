% Options for packages loaded elsewhere
\PassOptionsToPackage{unicode}{hyperref}
\PassOptionsToPackage{hyphens}{url}
%
\documentclass[
]{article}
\usepackage{amsmath,amssymb}
\usepackage{iftex}
\ifPDFTeX
  \usepackage[T1]{fontenc}
  \usepackage[utf8]{inputenc}
  \usepackage{textcomp} % provide euro and other symbols
\else % if luatex or xetex
  \usepackage{unicode-math} % this also loads fontspec
  \defaultfontfeatures{Scale=MatchLowercase}
  \defaultfontfeatures[\rmfamily]{Ligatures=TeX,Scale=1}
\fi
\usepackage{lmodern}
\ifPDFTeX\else
  % xetex/luatex font selection
\fi
% Use upquote if available, for straight quotes in verbatim environments
\IfFileExists{upquote.sty}{\usepackage{upquote}}{}
\IfFileExists{microtype.sty}{% use microtype if available
  \usepackage[]{microtype}
  \UseMicrotypeSet[protrusion]{basicmath} % disable protrusion for tt fonts
}{}
\makeatletter
\@ifundefined{KOMAClassName}{% if non-KOMA class
  \IfFileExists{parskip.sty}{%
    \usepackage{parskip}
  }{% else
    \setlength{\parindent}{0pt}
    \setlength{\parskip}{6pt plus 2pt minus 1pt}}
}{% if KOMA class
  \KOMAoptions{parskip=half}}
\makeatother
\usepackage{xcolor}
\usepackage[margin=1in]{geometry}
\usepackage{graphicx}
\makeatletter
\def\maxwidth{\ifdim\Gin@nat@width>\linewidth\linewidth\else\Gin@nat@width\fi}
\def\maxheight{\ifdim\Gin@nat@height>\textheight\textheight\else\Gin@nat@height\fi}
\makeatother
% Scale images if necessary, so that they will not overflow the page
% margins by default, and it is still possible to overwrite the defaults
% using explicit options in \includegraphics[width, height, ...]{}
\setkeys{Gin}{width=\maxwidth,height=\maxheight,keepaspectratio}
% Set default figure placement to htbp
\makeatletter
\def\fps@figure{htbp}
\makeatother
\setlength{\emergencystretch}{3em} % prevent overfull lines
\providecommand{\tightlist}{%
  \setlength{\itemsep}{0pt}\setlength{\parskip}{0pt}}
\setcounter{secnumdepth}{-\maxdimen} % remove section numbering
\newlength{\cslhangindent}
\setlength{\cslhangindent}{1.5em}
\newlength{\csllabelwidth}
\setlength{\csllabelwidth}{3em}
\newlength{\cslentryspacingunit} % times entry-spacing
\setlength{\cslentryspacingunit}{\parskip}
\newenvironment{CSLReferences}[2] % #1 hanging-ident, #2 entry spacing
 {% don't indent paragraphs
  \setlength{\parindent}{0pt}
  % turn on hanging indent if param 1 is 1
  \ifodd #1
  \let\oldpar\par
  \def\par{\hangindent=\cslhangindent\oldpar}
  \fi
  % set entry spacing
  \setlength{\parskip}{#2\cslentryspacingunit}
 }%
 {}
\usepackage{calc}
\newcommand{\CSLBlock}[1]{#1\hfill\break}
\newcommand{\CSLLeftMargin}[1]{\parbox[t]{\csllabelwidth}{#1}}
\newcommand{\CSLRightInline}[1]{\parbox[t]{\linewidth - \csllabelwidth}{#1}\break}
\newcommand{\CSLIndent}[1]{\hspace{\cslhangindent}#1}
\usepackage{float}
\usepackage{booktabs}
\usepackage{longtable}
\usepackage{array}
\usepackage{multirow}
\usepackage{wrapfig}
\usepackage{colortbl}
\usepackage{pdflscape}
\usepackage{tabu}
\usepackage{threeparttable}
\usepackage{threeparttablex}
\usepackage[normalem]{ulem}
\usepackage{makecell}
\usepackage{xcolor}
\ifLuaTeX
  \usepackage{selnolig}  % disable illegal ligatures
\fi
\IfFileExists{bookmark.sty}{\usepackage{bookmark}}{\usepackage{hyperref}}
\IfFileExists{xurl.sty}{\usepackage{xurl}}{} % add URL line breaks if available
\urlstyle{same}
\hypersetup{
  pdftitle={Investigation of the role of environmental variables on the occurrence of macroinvertebrates in Austria using multivariate statistical methods.},
  pdfauthor={MD. Dider Hossain (218203108)},
  hidelinks,
  pdfcreator={LaTeX via pandoc}}

\title{Investigation of the role of environmental variables on the
occurrence of macroinvertebrates in Austria using multivariate
statistical methods.}
\author{MD. Dider Hossain (218203108)}
\date{2025-06-10}

\begin{document}
\maketitle

\includegraphics{C:/Advance Data Analysis/unilogo.png}

\hypertarget{introduction}{%
\section{Introduction}\label{introduction}}

The species composition of various aquatic species is not the same in
all rivers or streams due to changes in some environmental parameters
and geographic factors (Li et al. 2012). Stream biodiversity is affected
by a combination of different environmental variables. Some of these
factors include deforestation, agricultural management, and excessive
use of hazardous fertilizers. (Ferreira et al. 2014). In addition, the
dam building in the river changes the flow of water, and the
accumulation of river water changes the physico-chemical features of the
water structure. These types of construction works also alter the
composition of the benthic macroinvertebrates population. (Ogbeibu and
Oribhabor 2002). Assessing habitat condition and water quality is
critical because it helps us understanding the composition of aquatic
communities (Ferreira et al. 2014). The presence of diverse
macroinvertebrates heavily influenced by the internal and exterior
circumstances of a stream (Li et al. 2012). For this reason,
macroinvertebrates species are frequently utilized to detect
environmental changes or evaluate environmental quality. Furthermore,
understanding the relationship among benthic macroinvertebrates, water
chemistry, and habitat conditions is critical for stream conservation.
(Ferreira et al. 2014). Different macroinvertebrates have different
habitat preferences, resulting in the diversity of macroinvertebrates in
streams (Li et al. 2012). The research also states that some
environmental parameters affect the abundance and distribution of
macroinvertebrates species, based on their tolerance to certain
environmental variables or their preference for certain habitat
requirements. According to Ferreira et al. (2014), temperature,
turbidity, and habitat loss can all be worsened when riparian vegetation
is removed or destroyed. The study also added that the conversion of
natural vegetation to agricultural land or cropland can have an impact
on water quality, increasing siltation below the water system, and
deteriorate habitat condition. In addition, a significant influence of
altitude and catchment properties on the composition of aquatic
macroinvertebrates was found in several lakes in Italy, Switzerland and
Austria (Füreder et al. 2006). Species use several techniques to adapt
to harsh environmental conditions. The functional characteristics of
species are the result of a long history of adaptation. (Choler 2005).
The feature of life cycle, adaptation technique to the harsh
environmental condition, inherent physiological characteristic of an
organism are the example of biological trait of a species. Feeding
behavior, reproductive technique, lifespan, maximum body size, and so
forth are examples of biological features of a species. These biological
characteristics determine their ability to overcome environmental
disturbances and adapt to the consequences of environmental changes. The
application of trait-based statistical methods has greatly expanded our
understanding of the effects of anthropogenic pressures on aquatic
ecosystems and revealed the relationships between species and their
environments. (Calapez et al. 2021). The primary objective of this paper
was to determine the effects of environmental factors on the abundance
of specific macroinvertebrates species and on the macroinvertebrates
community as a whole. We also attempted to ascertain how the biological
features of benthic macroinvertebrates respond to environmental factors?

\hypertarget{methodology}{%
\section{Methodology}\label{methodology}}

Data were obtained from the WISER (Water Integrative Systems to assess
Ecological status and Recovery) website
(\textbf{\href{http://www.wiser.eu/results/method-database/detail.php?id=49\&qst=country\%5B\%5D\%3DAustria\%26category\%5B\%5D\%3DRivers\%26bqe\%5B\%5D\%3DBenthic\%2520Invertebrates}{link}}).
The data are Water Framework Directive monitoring data collected at 198
different sites in Austria. The data collection method was developed by
BOKU- Institut für Hydrobiologie und Gewässermanagement, Arbeitsgruppe
Benthosökologie und Gewässerbewertung. Multiple habitats are sampled in
the most important habitats in proportion to their presence within a
sampling reach. Samples were collected from all habitat categories at
the sampling site with a minimum coverage of 5\%. ``Sampling unit'' is a
stationary sampling method that involves situating the net and
disrupting the substrate in a quadratic region equal to the frame-size
upstream of the net (0.25 x 0.25 m). Depending on the compactness of the
substrate, sediments were disturbed to a depth of 15-20 cm (where
possible). Hand net, Surber or Hess sampler were used to collect
macroinvertebrates species. Hand-net was 25 x 25 cm, mesh size 500 µm,
length of net was minimum 1m.

Macroinvertebrates species abundance data per site and environmental
variable data per site were merged by sample\_id. In this way, we
extracted only 327 sampling units out of 536 sampling units. Statistical
analyses were then performed on these selected data for
macroinvertebrates species abundance data and environmental variables
data. At first glance, a pairwise scatter plot was examined for our six
environmental variables and their relationships. In \textbf{Figure
1}(appendix section), we can see the distribution of the different data
points for all environmental variables. From the inspection of our data,
it was found that temperature data was balanced or normally distributed.
But, other variables such as catchment area, altitude, maximum depth,
conductivity, pH variables have skewed data. Since these variables each
have large outliers, log transformation was performed for those
variables. Data transformation for non-normal data is necessary to
stabilize the variance and increase the efficiency of data analysis
(Piepho 2009). To determine the highest and lowest species richness of
the 327 sampling sites, species richness was calculated using the
specnumber() function from the Vegan packages, and the Shannon-Weaver
index was also calculated as a diversity measure. To estimate the role
of environmental variables on the abundance of \emph{Haliplus} sp., a
multivariate generalized linear model (GLM) was conducted, and a model
was selected by a multiple stepwise backward regression model based on
the Akaike information criterion (AIC) value. Analysis of variance
inflation factor (VIF) was performed to test the problem of
multicollinearity among our environmental variables. From the VIF values
of each environmental variable, it found that the VIF values are below
the threshold \textbf{(VIF \textless{} 4)}, which means that we can rely
on our regression results based on these variables and that these
significant variables do not have a multicollinearity problem (Dormann
et al. 2013).

In addition, a canonical correspondence analysis (CCA) was conducted to
determine the effects of key environmental variables on whole aquatic
macroinvertebrates community abundance. In the case of CCA, a step()
function was used where a full model and a null model were defined. In
our case, we fitted a full model that included all of our predictor
variables, and the null model included an intercept that did not include
a predictor variable. Backward model selection, which is a part of
stepwise logistic regression, was also performed to identify the
best-fit model with significant environmental variables based on a lower
AIC value, where a lower AIC value indicates the best fit of the model.
According to Gareth et al. (2013) \& Bruce and Bruce (2017), backward
model selection can be applied when the number of samples is larger than
the number of predictors or continuous variables. In our case, 327
sampling units and six different predictor variables were considered
based on abundance data for 50 macroinvertebrates species, which means
that the sample size of individual taxa is larger than the explanatory
or continuous variables. After stepwise elimination of the predictor
variables, a VIF test was also performed to check for the
multicollinearity problem of our final four significant variables. The
permutation test was also performed on the final CCA model to evaluate
the significance of the fitted models. In this case, the species data
were permuted and the model was fitted again, which happened 999 times.
Permutational multivariate analysis of variance (PERMANOVA) was
performed with 999 simulations for the fitted model to determine the
significant effects of each environmental variable on macroinvertebrates
species composition.

The RLQ analysis was performed to determine the relationship among
environmental variable, functional traits and species abundance. The
primary goal of this analysis is to link a sites-by-environmental
variables table (R) to a species-by-traits table (Q), with a
plot-by-species table (L) acting as a bridge between R and Q.
Theoretically, the L matrix is assumed to quantify the degree of
correlation between the R and Q matrices. Three distinct ordinations
were used for three different data table. All trait variables were
transformed as numeric variables to ease our statistical analysis.
Correspondence analysis was performed to the species composition table
(L) by using dudi.coa function. Since, our environmental data and traits
data are contained quantitative variables and thus principal component
analysis was performed to the environmental variable table (R) and
species trait table (Q) by using dudi.pca function. (Choler 2005).

\hypertarget{results}{%
\section{Results}\label{results}}

\hypertarget{species-richness-and-diversity-metrices}{%
\subsection{Species richness and diversity
metrices}\label{species-richness-and-diversity-metrices}}

The specnumber() function from the Vegan package was used to calculate
species richness at 327 sampling sites. The highest number of
macroinvertebrates species was found in ``FW31001277B918'' (249)
sampling site, which included a total of 32 species. Only one
macroinvertebrates species was found in sampling site
``FW73200617B918''(314). This result is confirmed by the Shannon-Weaver
index (H), according to which ``FW31001277B918'' and ``FW73200617B918''
sampling sites have the highest and lowest Shannon-Weaver index,
respectively. At 327 sampling sites, the mean macroinvertebrates species
richness was 10.35 and the standard deviation was 4.15.

\hypertarget{role-of-environmental-variable-on-individual-taxa-abundance}{%
\subsection{Role of Environmental variable on individual taxa
abundance}\label{role-of-environmental-variable-on-individual-taxa-abundance}}

\begin{figure}

{\centering \includegraphics{MOD3_Multivariate_MD_DIDER_218203108_files/figure-latex/unnamed-chunk-8-1} 

}

\caption{**Figure-1:** Correlation plot of continuous variables (Transformed dataset)}\label{fig:unnamed-chunk-8}
\end{figure}

\textbf{Figure-1} shows the relationship between different continuous
variables and their correlation coefficients. For example, at the
intersection of maximum depth and catchment area, maximum depth defines
the vertical axis and catchment area defines the horizontal axis where
upper triangle gives the correlation coefficient (Greenacre and
Primicerio 2014). It was found that maximum depth had the highest
correlation with catchment area (0.43) and conductivity had the second
highest correlation coefficient (0.26) with temperature. In addition,
maximum depth also correlated positively with pH.

The multivariate GLM method was applied to determine the significant
influence of environmental variables on the abundance data of
\emph{Haliplus} sp. All environmental variables except altitude were
included in the initial model. The VIF analysis showed that all
variables have VIF values less than 4, which means that we can rely on
our regression result and they have no multicollinearity problem. The
significant model was selected based on the AIC value. The final fitted
model consisted of two environmental variables (temperature and
conductivity) with the lowest AIC value (88.93) compared to the other
models created. From the final model, it appears that only two variables
have a significant effect on the species diversity of \emph{Haliplus}
species out of 5 variables. The p-value of the final fitted model was
less than 0.05 (p-value = 2.11e-06), indicating that the model was
significant. It was also calculated that conductivity had the largest
significant effect on the abundance of \emph{Haliplus} sp. (p = 0.00109)
and temperature had a less significant effect (p = 0.03099) on the
abundance of \emph{Haliplus} species. It was cleared that the diversity
of \emph{Haliplus} sp. increased as temperature and conductivity
increased (\textbf{Figure-2}).

\begin{figure}

{\centering \includegraphics{MOD3_Multivariate_MD_DIDER_218203108_files/figure-latex/unnamed-chunk-15-1} 

}

\caption{**Figure-2**: Effect of environmental variables on abundance of *Haliplus* sp. (0 indicates absence of *Haliplus* sp. and 1 indicates presence of *Haliplus* sp. for the entire sampling site.)}\label{fig:unnamed-chunk-15}
\end{figure}

\hypertarget{effect-of-environmental-variable-on-whole-macroinvertebrate-species-composition}{%
\subsection{Effect of Environmental variable on whole macroinvertebrate
species
composition}\label{effect-of-environmental-variable-on-whole-macroinvertebrate-species-composition}}

The final fitted model to determine the effects of environmental
variables on species diversity of the whole macroinvertebrates community
estimated total variance in our species community matrix to be 3.37 and
0.32 variance explained by four significant environmental variables. To
ensure model validity, regression residuals were also examined.
(Ferreira et al. 2014). The cumulative proportion for CCA1 and CCA2 was
0.86, which means that these two axes created by CCA can explain 86\% of
the relationships between macroinvertebrates species composition and
environmental variables. Between the two CCA axes, the largest variation
was explained by CCA1, which was about 70\%, and CCA2 explained 16\% of
the total variability in the relationship between species composition
and environmental variables. The proportion value (from the summary
statistics of the fitted model) shows that 10\% of the variance in
species distribution can be explained by the variables temperature,
catchment area, altitude and conductivity. On the other hand, the
unconstrained axis (residuals) can explain 90\% of the variance in the
distribution of macroinvertebrates species. From the eigenvalue, we
found that CCA1 and CCA2 axis explained 0.28 variance in the species
distribution data sets (\textbf{Table-1}).

\begin{verbatim}
## Warning: 'xfun::attr()' is deprecated.
## Use 'xfun::attr2()' instead.
## See help("Deprecated")

## Warning: 'xfun::attr()' is deprecated.
## Use 'xfun::attr2()' instead.
## See help("Deprecated")

## Warning: 'xfun::attr()' is deprecated.
## Use 'xfun::attr2()' instead.
## See help("Deprecated")

## Warning: 'xfun::attr()' is deprecated.
## Use 'xfun::attr2()' instead.
## See help("Deprecated")

## Warning: 'xfun::attr()' is deprecated.
## Use 'xfun::attr2()' instead.
## See help("Deprecated")

## Warning: 'xfun::attr()' is deprecated.
## Use 'xfun::attr2()' instead.
## See help("Deprecated")

## Warning: 'xfun::attr()' is deprecated.
## Use 'xfun::attr2()' instead.
## See help("Deprecated")

## Warning: 'xfun::attr()' is deprecated.
## Use 'xfun::attr2()' instead.
## See help("Deprecated")

## Warning: 'xfun::attr()' is deprecated.
## Use 'xfun::attr2()' instead.
## See help("Deprecated")

## Warning: 'xfun::attr()' is deprecated.
## Use 'xfun::attr2()' instead.
## See help("Deprecated")

## Warning: 'xfun::attr()' is deprecated.
## Use 'xfun::attr2()' instead.
## See help("Deprecated")

## Warning: 'xfun::attr()' is deprecated.
## Use 'xfun::attr2()' instead.
## See help("Deprecated")

## Warning: 'xfun::attr()' is deprecated.
## Use 'xfun::attr2()' instead.
## See help("Deprecated")

## Warning: 'xfun::attr()' is deprecated.
## Use 'xfun::attr2()' instead.
## See help("Deprecated")

## Warning: 'xfun::attr()' is deprecated.
## Use 'xfun::attr2()' instead.
## See help("Deprecated")

## Warning: 'xfun::attr()' is deprecated.
## Use 'xfun::attr2()' instead.
## See help("Deprecated")
\end{verbatim}

\begin{table}
\centering
\caption{\label{tab:unnamed-chunk-21}**Table-1** : Statistical summary of CCA for environmental variables and macroinvertebrate species}
\centering
\begin{tabu} to \linewidth {>{\raggedright}X>{\raggedleft}X>{\raggedleft}X>{\raggedleft}X>{\raggedleft}X}
\hline
  & Df & ChiSquare & F & Pr(>F)\\
\hline
CCA1 & 1 & 0.2303117 & 24.329287 & 0.001\\
\hline
CCA2 & 1 & 0.0509905 & 5.386451 & 0.001\\
\hline
CCA3 & 1 & 0.0269507 & 2.846970 & 0.001\\
\hline
CCA4 & 1 & 0.0182049 & 1.923098 & 0.004\\
\hline
Residual & 322 & 3.0481937 & NA & NA\\
\hline
\end{tabu}
\end{table}

A permutation test of the fitted model was performed where the data was
shuffling for 999 times and refitting the model. To find out whether our
final model is significant or not, a permutation test was performed. The
permutation test showed that the model consisted of 4 variables and the
chi-square value was 0.32 for the four environmental variables. The
degrees of freedom for the residual were 323 and the chi-square value
for the residual was 3.04. And the F-statistic was 8.16 for our fitted
model which was high. In addition, the p value for the fitted model was
0.001, which means that 99.99\% of the permuted values had an F
statistic that was smaller, and only about 0.1\% of the permuted F
statistics for the calculated model were larger than the observed
statistics, which also indicated that our model was significant
(\textbf{Figure-2}(appendix section)). And the adjusted R-square value
for the fitted model was 8.5\%. The PERMANOVA test was performed to
determine the significance level for each environmental variable. The
PERMANOVA test showed that all four variables have a highly significant
effect (p = 0.001) on the macroinvertebrates species composition. But
temperature has the highest F-statistic value, followed by elevation,
catchment area and conductivity. (\textbf{Table-2})

\begin{verbatim}
## Warning: 'xfun::attr()' is deprecated.
## Use 'xfun::attr2()' instead.
## See help("Deprecated")

## Warning: 'xfun::attr()' is deprecated.
## Use 'xfun::attr2()' instead.
## See help("Deprecated")

## Warning: 'xfun::attr()' is deprecated.
## Use 'xfun::attr2()' instead.
## See help("Deprecated")

## Warning: 'xfun::attr()' is deprecated.
## Use 'xfun::attr2()' instead.
## See help("Deprecated")

## Warning: 'xfun::attr()' is deprecated.
## Use 'xfun::attr2()' instead.
## See help("Deprecated")

## Warning: 'xfun::attr()' is deprecated.
## Use 'xfun::attr2()' instead.
## See help("Deprecated")

## Warning: 'xfun::attr()' is deprecated.
## Use 'xfun::attr2()' instead.
## See help("Deprecated")

## Warning: 'xfun::attr()' is deprecated.
## Use 'xfun::attr2()' instead.
## See help("Deprecated")

## Warning: 'xfun::attr()' is deprecated.
## Use 'xfun::attr2()' instead.
## See help("Deprecated")

## Warning: 'xfun::attr()' is deprecated.
## Use 'xfun::attr2()' instead.
## See help("Deprecated")

## Warning: 'xfun::attr()' is deprecated.
## Use 'xfun::attr2()' instead.
## See help("Deprecated")

## Warning: 'xfun::attr()' is deprecated.
## Use 'xfun::attr2()' instead.
## See help("Deprecated")

## Warning: 'xfun::attr()' is deprecated.
## Use 'xfun::attr2()' instead.
## See help("Deprecated")

## Warning: 'xfun::attr()' is deprecated.
## Use 'xfun::attr2()' instead.
## See help("Deprecated")

## Warning: 'xfun::attr()' is deprecated.
## Use 'xfun::attr2()' instead.
## See help("Deprecated")

## Warning: 'xfun::attr()' is deprecated.
## Use 'xfun::attr2()' instead.
## See help("Deprecated")
\end{verbatim}

\begin{table}
\centering
\caption{\label{tab:unnamed-chunk-25}**Table-2** : PERMANOVA test summary for the fitted model}
\centering
\begin{tabu} to \linewidth {>{\raggedright}X>{\raggedleft}X>{\raggedleft}X>{\raggedleft}X>{\raggedleft}X>{\raggedleft}X}
\hline
  & Df & SumOfSqs & R2 & F & Pr(>F)\\
\hline
temperature & 1 & 7.729626 & 0.1214666 & 50.629324 & 0.001\\
\hline
Cat\_Area & 1 & 1.620338 & 0.0254627 & 10.613269 & 0.001\\
\hline
Altitude & 1 & 4.006314 & 0.0629569 & 26.241498 & 0.001\\
\hline
Conductivity & 1 & 1.119503 & 0.0175923 & 7.332783 & 0.001\\
\hline
Residual & 322 & 49.160041 & 0.7725215 & NA & NA\\
\hline
Total & 326 & 63.635821 & 1.0000000 & NA & NA\\
\hline
\end{tabu}
\end{table}

The CCA1 axis exhibited the strongest inverse relation with temperature,
followed by conductivity and catchment area. As an environmental
gradient, the CCA1 axis depicts the decreasing characteristics of
temperature, conductivity, and catchment area. On the other hand,
altitude is positively correlated with CCA1 axis (\textbf{Table-3},
\textbf{Figure-3}).

\begin{verbatim}
## Warning: 'xfun::attr()' is deprecated.
## Use 'xfun::attr2()' instead.
## See help("Deprecated")

## Warning: 'xfun::attr()' is deprecated.
## Use 'xfun::attr2()' instead.
## See help("Deprecated")

## Warning: 'xfun::attr()' is deprecated.
## Use 'xfun::attr2()' instead.
## See help("Deprecated")

## Warning: 'xfun::attr()' is deprecated.
## Use 'xfun::attr2()' instead.
## See help("Deprecated")

## Warning: 'xfun::attr()' is deprecated.
## Use 'xfun::attr2()' instead.
## See help("Deprecated")

## Warning: 'xfun::attr()' is deprecated.
## Use 'xfun::attr2()' instead.
## See help("Deprecated")

## Warning: 'xfun::attr()' is deprecated.
## Use 'xfun::attr2()' instead.
## See help("Deprecated")

## Warning: 'xfun::attr()' is deprecated.
## Use 'xfun::attr2()' instead.
## See help("Deprecated")

## Warning: 'xfun::attr()' is deprecated.
## Use 'xfun::attr2()' instead.
## See help("Deprecated")

## Warning: 'xfun::attr()' is deprecated.
## Use 'xfun::attr2()' instead.
## See help("Deprecated")

## Warning: 'xfun::attr()' is deprecated.
## Use 'xfun::attr2()' instead.
## See help("Deprecated")

## Warning: 'xfun::attr()' is deprecated.
## Use 'xfun::attr2()' instead.
## See help("Deprecated")

## Warning: 'xfun::attr()' is deprecated.
## Use 'xfun::attr2()' instead.
## See help("Deprecated")

## Warning: 'xfun::attr()' is deprecated.
## Use 'xfun::attr2()' instead.
## See help("Deprecated")

## Warning: 'xfun::attr()' is deprecated.
## Use 'xfun::attr2()' instead.
## See help("Deprecated")

## Warning: 'xfun::attr()' is deprecated.
## Use 'xfun::attr2()' instead.
## See help("Deprecated")
\end{verbatim}

\begin{table}
\centering
\caption{\label{tab:unnamed-chunk-26}**Table-3** : Correlation between environmental variables and derived CCA axis}
\centering
\begin{tabu} to \linewidth {>{\raggedright}X>{\raggedright}X>{\raggedright}X>{\raggedleft}X>{\raggedleft}X>{\raggedleft}X}
\hline
  & score & label & CCA1 & CCA2 & CCA3\\
\hline
705 & biplot & temperature & -0.7708337 & 0.0871989 & 0.5951179\\
\hline
706 & biplot & Cat\_Area & -0.1467410 & -0.9541516 & 0.1936259\\
\hline
707 & biplot & Altitude & 0.8862472 & 0.1672755 & 0.1059058\\
\hline
708 & biplot & Conductivity & -0.7211071 & -0.1983873 & -0.6040314\\
\hline
\end{tabu}
\end{table}

On the CCA1 axis, individual taxa with high positive score included
Chironomidae (\emph{Cricotopus}(Paratrichocladius) \emph{rufiventris}),
Chironomidae (\emph{Corynoneura} sp.), Chironomidae
(\emph{Parametriocnemus stylatus}), Elmidae (\emph{Elmis} sp.), Baetidae
(\emph{Baetis rhodani}), Limoniidae (\emph{Hexatoma} sp.), Chironomidae
(\emph{Heleniella ornaticollis}), \emph{Hydrachnidia Gen.} sp.,
Sericostomatidae (\emph{Sericostoma personatum}) and the taxa with the
high negative score included Platycnemididae (\emph{Platycnemis
pennipes}), Valvatidae (\emph{Valvata piscinalis}), Naididae
(\emph{Psammoryctides barbatus}), Baetidae (\emph{Cloeon} (Cloeon)
\emph{dipterum}) , Naididae (\emph{Limnodrilus udekemianus}) ,
Chironomidae (\emph{Procladius} sp.), Chironomidae (\emph{Polypedilum}
(Tripodura) \emph{scalaenum} Gr.), Glossiphoniidae (\emph{Helobdella
stagnalis}), Aeshnidae (\emph{Anisoptera} Gen.~sp.), Chironomidae
(\emph{Polypedilum} (Polypedilum) \emph{albicorne}), Chironomidae
(\emph{Tanytarsus ejuncidus}). (\textbf{Figure-3})

Furthermore, CCA2 axis explained 16\% variability of the
macroinvertebrate abundance and environmental relationship. From the
axis score, temperature and altitude are positively correlated with the
CCA2 axis, while altitude had the highest positive correlation with the
CCA2 axis and the highest negative correlation found for the catchment
area with the CCA2 axis, followed by conductivity (\textbf{Table-3}).
For this reason, the CCA2 axis was plotted as a gradient with decreasing
catchment area and conductivity variables, and as a gradient with
increasing altitude and temperature variable (\textbf{Figure-3}).
Macroinvertebrates species with high positive score in the CCA2 axis
included Corixidae (\emph{Micronecta scholtzi}), Nepidae (\emph{Nepa
cinerea}), Chironomidae (\emph{Limnophyes} sp.), Aeshnidae
(\emph{Anisoptera} Gen.~sp.), Hydraenidae (\emph{Hydraena gracilis}
Ad.), Chironomidae (\emph{Microtendipes rydalensis}) and the species
with highest negative score include Platycnemididae (\emph{Platycnemis
pennipes}), Naididae (\emph{Psammoryctides albicola}), Limnephilidae
(\emph{Allogamus auricollis}), Hydroptilidae (\emph{Ithytrichia
lamellaris}), Naididae (\emph{Potamothrix bavaricus}), Chironomidae
(\emph{Cricotopus}(Paratrichocladius) \emph{rufiventris}) ,
\emph{Leptoceridae} Gen.~sp., Chironomidae (\emph{Tanytarsus
ejuncidus}). (\textbf{Figure-3})

\begin{verbatim}
## Warning: Using `size` aesthetic for lines was deprecated in ggplot2 3.4.0.
## i Please use `linewidth` instead.
## This warning is displayed once every 8 hours.
## Call `lifecycle::last_lifecycle_warnings()` to see where this warning was
## generated.
\end{verbatim}

\begin{figure}

{\centering \includegraphics{MOD3_Multivariate_MD_DIDER_218203108_files/figure-latex/unnamed-chunk-27-1} 

}

\caption{**Figure-3:** Species score and biplot score triplot on the basis of CCA (Species are indicated by the blue arrows and environmental variables are indicated by the red arrows.)}\label{fig:unnamed-chunk-27}
\end{figure}

\begin{figure}

{\centering \includegraphics{MOD3_Multivariate_MD_DIDER_218203108_files/figure-latex/unnamed-chunk-28-1} 

}

\caption{**Figure-4:** Site score and biplot score triplot on the basis of CCA (Sites are indicated by the black dots and environmental variables are indicated by the red arrows.)}\label{fig:unnamed-chunk-28}
\end{figure}

\hypertarget{response-of-biological-traits}{%
\subsection{Response of biological
traits}\label{response-of-biological-traits}}

\begin{figure}

{\centering \includegraphics{MOD3_Multivariate_MD_DIDER_218203108_files/figure-latex/unnamed-chunk-33-1} 

}

\caption{**Figure-5:** Distribution of environmental variables, species characteristics, and macroinvertebrate species along the first two RLQ axes (the horizontal axis corresponds to RLQ axis-1 and the vertical axis corresponds to RLQ axis-2).)}\label{fig:unnamed-chunk-33}
\end{figure}

The first axis of RLQ explained the most variability of the R and Q
matrix. On this axis, 98\% of the expected inertia was extracted, while
only 1.7\% was extracted on the second axis (From the summary statistics
of RLQ analysis). The total projected inertia described by these two
axes is 99.86\%, indicating that this two axis can explain the most of
the variability in our trait and environmental variable databases. The
correlation between the new site score (constrained by environmental
factors) and the new species score (constrained by species
characteristics) is minimized using RLQ analysis (Choler 2005). The
eigenvalue and covariance for the first axis were 0.44 and 0.66,
respectively. However, for the second axis, the eigenvalue and
covariance were 0.008 and 0.09, respectively, and thus lower than those
for the first axis. The correlation between new site score and species
score in RLQ axis1 was 0.285 and in RLQ axis2 was 0.053. \textbf{Figure
5} illustrates to understand the relationships among environmental
variables, species characteristics, and macroinvertebrate assemblage
along the first two RLQ axes. From this figure, we can see that the left
part of the RLQ axis1 represents higher conductivity and response traits
of \emph{Haliplus} sp is positively correlated with the conductivity.
But, the voltinism traits of the \emph{Haliplus} sp. is negatively
correlated with the conductivity. Moreover, feeding behavior and
oviposition traits of \emph{Asellus aquaticus} species are positively
correlated with the temperature and locomotion traits is negatively
correlated with the temperature. Voltinism traits of \emph{Elmis} sp. is
positively correlated with the altitude and response traits is
negatively correlated with the altitude. And locomotion traits of
\emph{Baetis} sp. is positively correlated with the pH and feeding
behavior, response traits are negatively correlated with the pH.
(\textbf{Figure-5})

\hypertarget{discussion}{%
\section{Discussion}\label{discussion}}

Macroinvertebrates distribution depended significantly (p-value
\textless0.05) on temperature, catchment area, elevation, and
conductivity (\textbf{Table-2}). A significant positive relationship was
found between macroinvertebrates abundance and altitude variables
(\textbf{Table-3}). A negative relationship was found for catchment area
and conductivity (\textbf{Table-3}). From the CCA analysis, temperature
is negatively correlated with the first CCA1 axis, which means that this
axis is interpreted as a gradient for the decrease of the temperature
variable (\textbf{Figure-3}). Some macro invertebrate for example,
Aeshnidae (\emph{Anisoptera} Gen.~sp.), Corixidae (\emph{Micronecta
scholtzi}), Nepidae (\emph{Nepa cinerea}), Chironomidae (\emph{Tanypus
punctipennis}), Baetidae (\emph{Baetis nexus}), Baetidae (\emph{Cloeon}
(Cloeon) \emph{dipterum}), Naididae (\emph{Limnodrilus udekemianus}),
Hydropsychidae (\emph{Hydropsyche bulbifera}), Gerridae (\emph{Gerris}
sp.) and more species are positively correlated with temperature,
indicating that these species are temperature-tolerant species. On the
other hand, macroinvertebrate species like Elmidae (\emph{Elmis} sp.),
Baetidae (\emph{Baetis} sp.), Hydrachnidiae (\emph{Hydrachnidia}
Gen.~sp.), Limoniidae (\emph{Hexatoma} sp.), Limnephilidae
(\emph{Allogamus auricollis}) are negatively correlated with
temperature, which means that the abundance of these species decreases
with increasing temperature. Temperature has a significant effect on the
respiration rate of benthic macroinvertebrates, and in this way
temperature controls oxygen availability to macroinvertebrates species
(Sharifinia, Imanpour Namin, and Bozorgi Makrani 2012). Altitude
correlated positively with both CCA axes and also had a significant
effect on invertebrate fauna. Some families of macroinvertebrates
species, e.g.~Sericostomatidae. (\emph{Sericostoma flavicorne}),
Chironomidae (\emph{Heleniella ornaticollis}), Hydraenidae
(\emph{Hydraena gracilis} Ad.), Naididae (\emph{Pristina} sp.),
Chironomidae (\emph{Corynoneura} sp.), Chironomidae
(\emph{Parametriocnemus stylatus}), Limoniidae (\emph{Hexatoma} sp.),
Limnephilidae (\emph{Allogamus auricollis}) were increased with
increasing of altitude (\textbf{Figure-3}). From the correlation plot,
it can be seen that environmental variables such as temperature and
conductivity decrease with increasing altitude (\textbf{Figure-1}).
Temperature and conductivity decrease at higher elevations. Due to the
longer freeze period, food availability, growth rate and time to
complete the biological cycle are reduced. For this reason, some
macroinvertebrates are observed in large numbers due to their
adaptability to these extreme environmental conditions. (Kasangaki,
Chapman, and Balirwa 2008). A study conducted by Füreder et al. (2006)
examined the effects of altitude and catchment characteristics on
macroinvertebrate communities in several lakes in Europe (Italy,
Switzerland, Austria). It was found that the abundance of
macroinvertebrates per sample increased with increasing elevation up to
a certain altitude, after that limit the abundance of benthic
macroinvertebrates decreased. Conductivity increases abundance of some
family of macroinvertebrate species such as Haliplidae (\emph{Haliplus}
sp.), Naididae (\emph{Psammoryctides barbatus}), \emph{Leptoceridae}
Gen.~sp., Chironomidae (\emph{Tanytarsus ejuncidus}), \emph{Zygoptera}
Gen.~sp., Naididae (\emph{Potamothrix bavaricus}), Platycnemididae
(\emph{Platycnemis pennipes}) and with increasing of catchment area
Hydroptilidae(\emph{Ithytrichia lamellaris}) species abundance was also
increased. In addition, Chironomidae (\emph{Thienemanniella} sp.),
Chironomidae (\emph{Corynoneura} sp.), Naididae (\emph{Pristina} sp.),
Chironomidae (\emph{Heleniella ornaticollis}), Chironomidae
(\emph{Microtendipes rydalensis}), Hydraenidae (\emph{Hydraena gracilis}
Ad.) species abundances are reduced with increasing temperature and
conductivity (\textbf{Figure-3}). When agricultural runoff containing
harmful fertilizers and industrial waste water enters waterways, the
conductivity of the stream water increases. The introduction of salt
into the stream water also increases the conductivity of the stream
water. (Encina-Montoya et al. 2020) The researchers investigated the
impact of salt usage in fish farms on macroinvertebrates drift in a
freshwater stream and discovered that increasing conductivity had an
influence on macroinvertebrate drift rates and the family of
macroinvertebrates included Naididae, Chironomidae. A study by
Miserendino (2001) examined the relationship between macroinvertebrates
and environmental variables and found also same result like us that some
family of macroinvertebrates abundance increased with increasing
conductivity, temperature, and elevation. The response of biological
traits of Haliplidae (\emph{Haliplus} sp.), Baetidae (\emph{Baetis}
sp.), Elmidae (\emph{Elmis} sp.), and Asellidae (\emph{Asellus
aquaticus}) to environmental variables was examined using RLQ analysis.
RLQ analysis revealed that conductivity had a positive effect on the
response trait and abundance of Haliplidae (\emph{Haliplus} sp.) family
of macroinvertebrates. Temperature had positive impacts on the feeding
behavior and increasing pre- and post-laying behavior of Asellidae
(\emph{Asellus aquaticus}) species (Lancaster and Downes 2013), which
means that as temperature increases, so do egg-laying opportunities and
abundance for macroinvertebrates of the family Asellidae. The number of
breeds raised per year or Voltinism of Elmidae (\emph{Elmis} sp.) is
increased with increasing altitude. \emph{Elmis} sp. referred as
univoltine races and it was also cleared that the voltinism is very
sensitive to temperature and conductivity for \emph{Elmis} sp.
(\textbf{Figure-5}). In addition, the swimming ability and body size of
the Baetidae family (\emph{Baetis} sp.) were found to increase with
increasing stream water pH which means this species is pH-tolerant
species.

\hypertarget{appendix}{%
\section{Appendix}\label{appendix}}

\begin{figure}

{\centering \includegraphics{MOD3_Multivariate_MD_DIDER_218203108_files/figure-latex/unnamed-chunk-40-1} 

}

\caption{ **Figure-1:** Pairwise scatterplot of continuous variables for macroinvertebrate data}\label{fig:unnamed-chunk-40}
\end{figure}

\begin{figure}

{\centering \includegraphics{MOD3_Multivariate_MD_DIDER_218203108_files/figure-latex/unnamed-chunk-41-1} 

}

\caption{ **Figure-2:** Density plot for permutation test of final fitted CCA model}\label{fig:unnamed-chunk-41}
\end{figure}

\begin{figure}

{\centering \includegraphics{MOD3_Multivariate_MD_DIDER_218203108_files/figure-latex/unnamed-chunk-42-1} 

}

\caption{**Figure-3**: Effect of temperature on abundance of macroinvertebrate species (0 indicates absence and 1 indicates presence of macroinvertebrates for the entire sampling site.)}\label{fig:unnamed-chunk-42}
\end{figure}

\begin{figure}

{\centering \includegraphics{MOD3_Multivariate_MD_DIDER_218203108_files/figure-latex/unnamed-chunk-43-1} 

}

\caption{**Figure-4**: Effect of conductivity on abundance of macroinvertebrate species (0 indicates absence and 1 indicates presence of macroinvertebrates for the entire sampling site.)}\label{fig:unnamed-chunk-43}
\end{figure}
\begin{figure}

{\centering \includegraphics{MOD3_Multivariate_MD_DIDER_218203108_files/figure-latex/unnamed-chunk-44-1} 

}

\caption{**Figure-5**: Effect of altitude on abundance of macroinvertebrate species (0 indicates absence and 1 indicates presence of macroinvertebrates for the entire sampling site.)}\label{fig:unnamed-chunk-44}
\end{figure}

\hypertarget{reference}{%
\section*{Reference}\label{reference}}
\addcontentsline{toc}{section}{Reference}

\hypertarget{refs}{}
\begin{CSLReferences}{1}{0}
\leavevmode\vadjust pre{\hypertarget{ref-bruce2017practical}{}}%
Bruce, P, and A Bruce. 2017. {``Practical Statistics for Data
Scientists, OREILLY Sebastopol, California-USA.''}

\leavevmode\vadjust pre{\hypertarget{ref-calapez2021influence}{}}%
Calapez, Ana Raquel, Sónia RQ Serra, Rui Rivaes, Francisca C Aguiar, and
Maria João Feio. 2021. {``Influence of River Regulation and Instream
Habitat on Invertebrate Assemblage'structure and Function.''}
\emph{Science of The Total Environment} 794: 148696.

\leavevmode\vadjust pre{\hypertarget{ref-choler2005consistent}{}}%
Choler, Philippe. 2005. {``Consistent Shifts in Alpine Plant Traits
Along a Mesotopographical Gradient.''} \emph{Arctic, Antarctic, and
Alpine Research} 37 (4): 444--53.

\leavevmode\vadjust pre{\hypertarget{ref-dormann2013collinearity}{}}%
Dormann, Carsten F, Jane Elith, Sven Bacher, Carsten Buchmann, Gudrun
Carl, Gabriel Carré, Jaime R Garcı́a Marquéz, et al. 2013.
{``Collinearity: A Review of Methods to Deal with It and a Simulation
Study Evaluating Their Performance.''} \emph{Ecography} 36 (1): 27--46.

\leavevmode\vadjust pre{\hypertarget{ref-encina2020relationship}{}}%
Encina-Montoya, Francisco, Luz Boyero, Alan M Tonin, Marı́a Fernanda
Aguayo, Carlos Esse, Rolando Vega, Francisco Correa-Araneda, Carlos
Oberti, and Jorge Nimptsch. 2020. {``Relationship Between Salt Use in
Fish Farms and Drift of Macroinvertebrates in a Freshwater Stream.''}

\leavevmode\vadjust pre{\hypertarget{ref-ferreira2014importance}{}}%
Ferreira, Wander R, Raphael Ligeiro, Diego R Macedo, Robert M Hughes,
Philip R Kaufmann, Leandro G Oliveira, and Marcos Callisto. 2014.
{``Importance of Environmental Factors for the Richness and Distribution
of Benthic Macroinvertebrates in Tropical Headwater Streams.''}
\emph{Freshwater Science} 33 (3): 860--71.

\leavevmode\vadjust pre{\hypertarget{ref-fureder2006macroinvertebrate}{}}%
Füreder, L, R Ettinger, A Boggero, B Thaler, and H Thies. 2006.
{``Macroinvertebrate Diversity in Alpine Lakes: Effects of Altitude and
Catchment Properties.''} \emph{Hydrobiologia} 562 (1): 123--44.

\leavevmode\vadjust pre{\hypertarget{ref-gareth2013introduction}{}}%
Gareth, James, Witten Daniela, Hastie Trevor, and Tibshirani Robert.
2013. \emph{An Introduction to Statistical Learning: With Applications
in r}. Spinger.

\leavevmode\vadjust pre{\hypertarget{ref-greenacre2014multivariate}{}}%
Greenacre, Michael, and Raul Primicerio. 2014. \emph{Multivariate
Analysis of Ecological Data}. Fundacion BBVA.

\leavevmode\vadjust pre{\hypertarget{ref-kasangaki2008land}{}}%
Kasangaki, Aventino, Lauren J Chapman, and John Balirwa. 2008. {``Land
Use and the Ecology of Benthic Macroinvertebrate Assemblages of
High-Altitude Rainforest Streams in Uganda.''} \emph{Freshwater Biology}
53 (4): 681--97.

\leavevmode\vadjust pre{\hypertarget{ref-lancaster2013aquatic}{}}%
Lancaster, Jill, and Barbara J Downes. 2013. \emph{Aquatic Entomology}.
OUP Oxford.

\leavevmode\vadjust pre{\hypertarget{ref-li2012relationships}{}}%
Li, Fengqing, Namil Chung, MI-JUNG BAE, YONG-SU KWON, and YOUNG-SEUK
PARK. 2012. {``Relationships Between Stream Macroinvertebrates and
Environmental Variables at Multiple Spatial Scales.''} \emph{Freshwater
Biology} 57 (10): 2107--24.

\leavevmode\vadjust pre{\hypertarget{ref-miserendino2001macroinvertebrate}{}}%
Miserendino, Maria Laura. 2001. {``Macroinvertebrate Assemblages in
Andean Patagonian Rivers and Streams: Environmental Relationships.''}
\emph{Hydrobiologia} 444: 147--58.

\leavevmode\vadjust pre{\hypertarget{ref-ogbeibu2002ecological}{}}%
Ogbeibu, AE, and BJ Oribhabor. 2002. {``Ecological Impact of River
Impoundment Using Benthic Macro-Invertebrates as Indicators.''}
\emph{Water Research} 36 (10): 2427--36.

\leavevmode\vadjust pre{\hypertarget{ref-piepho2009data}{}}%
Piepho, Hans-Peter. 2009. {``Data Transformation in Statistical Analysis
of Field Trials with Changing Treatment Variance.''} \emph{Agronomy
Journal} 101 (4): 865--69.

\leavevmode\vadjust pre{\hypertarget{ref-sharifinia2012benthic}{}}%
Sharifinia, M, J Imanpour Namin, and A Bozorgi Makrani. 2012. {``Benthic
Macroinvertabrate Distribution in Tajan River Using Canonical
Correspondence Analysis.''}

\end{CSLReferences}

\end{document}
